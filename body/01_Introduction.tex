\section{Introduction}

In recent years, the popularity of platforms for the storage and transmission of video content has enabled a massive volume of video data production and consumption.
%%
As an example, in 2019, more than one billion hours of YouTube videos were watched per day.\footnote{https://kinsta.com/blog/youtube-stats/}
%%
Let us define relevant people present in a video file as \emph{actors}.
%%
Generating metadata with the identity information of the actors present and their localization both in video frames~(spatial) and timeline~(temporal) can facilitate video indexing, retrieval, recommendation and a series of other tasks.
%%

Face detection methods have been attracting the attention of researchers for more than two decades~\cite{survey66}. 
Nowadays, it is used for surveillance, video analytics systems, smart shopping, automatic face tagging in photo collections, investigative tools that search for identities in social networks based on face images, and thousands of other applications in our daily lives.
For instance, Facer~\cite{hazelwood2018applied} is the Facebook's face detection and recognition framework; given a photograph, it first detects all the faces, and then runs a deep model to determine the likelihood of that face belonging to one of the top-N user friends.
This allows Facebook to suggest which friends the user might want to tag within the uploaded photographs. 


%%
This dissertation investigates a method for this spatiotemporal localization and a series of applications that such a method enables. We investigate its applicability in 2D traditional videos and 360-video, which have been gaining attention in recent years~\cite{mendes2020authoring}.

%%
The remainder of this dissertation proposal is structured as follows.


